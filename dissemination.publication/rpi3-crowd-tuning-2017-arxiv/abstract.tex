Developing efficient software and hardware has never been harder whether it is
for a tiny IoT device or an Exascale supercomputer.
%
Apart from the ever growing design and optimization complexity, there exist even
more fundamental problems such as lack of interdisciplinary knowledge required
for effective software/hardware co-design, and a growing technology transfer
gap between academia and industry.
%

We introduce our new educational initiative to tackle these problems by
developing Collective Knowledge (CK), a unified experimental framework for
computer systems research and development.
%
We use CK to teach the community how to make their research artifacts and
experimental workflows portable, reproducible, customizable and reusable while
enabling sustainable R\&D and facilitating technology transfer.
%
We also demonstrate how to redesign multi-objective autotuning and machine learning
as a portable and extensible CK workflow.
%
Such workflows enable researchers to experiment with different applications,
data sets and tools; crowdsource experimentation across diverse platforms;
share experimental results, models, visualizations; gradually expose more
design and optimization choices using a simple JSON API; and ultimately build
upon each other's findings.
%

As the first practical step, we have implemented customizable 
compiler autotuning, crowdsourced optimization 
of diverse workloads across Raspberry Pi 3 devices,
reduced the execution time and code size by up to 40\%,
and applied machine learning to predict optimizations.
%
We hope such approach will help teach students how to build 
upon each others' work to enable efficient and self-optimizing
software/hardware/model stack for emerging workloads.
